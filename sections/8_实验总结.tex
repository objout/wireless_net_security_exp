\documentclass[../main.tex]{subfiles}
\graphicspath{{\subfix{../figures/}}}

\begin{document}
\section{实验总结}
\subsection{小组成员}
按照学号升序排序:
\begin{table}[H]
  \caption{小组成员}
  \begin{center}
    \begin{tabular}[c]{ll}
      \hline
      姓名 & 学号 \\
      \hline
      方赞 & 102102114 \\
      许诚龙 & 102102118 \\
      张超祥(组长) & 102102132 \\
      胡嘉鑫 & 102102145 \\
      \hline
    \end{tabular}
  \end{center}
\end{table}
%
\subsection{方赞}
本次Deauth Flood攻防实验让我深入了解了这种攻击方式的原理、过程及防御策略。
通过实施攻击实验,我们可以更好地了解攻击者的手段和方法,从而制定出更有效的防御策略。
同时,在防御过程中,我们也需要不断学习和掌握新的技术和方法,以应对日益复杂的网络安全威胁。
%
\subsection{许诚龙}
通过本次实验,我深入了解了Deauth攻击的原理、实现方式及其影响。
实验结果表明,Deauth攻击具有较高的隐蔽性和破坏性,能够对无线网络造成严重的安全威胁。
因此,我们需要加强对无线网络安全的防护和监管,提高用户的网络安全意识,共同维护网络空间的安全稳定。
%
\subsection{张超祥}
在本次实验中,我们深入研究了Deauth Flood 攻击及其防御方法。
通过使用开源项目esp8266\_deauther和相关工具,我们成功模拟了Deauth攻击,并对其原理和影响进行了详细分析。
学习到了一种攻击方式的原理,更加加深了我对网络安全的认识。
%
\subsection{胡嘉鑫}
通过本次实验,我了解了无线网络通信的原理,明白了无线网络中存在的不安全性。
同时也意识到:网络安全任重而道远,需我辈青年增强网络空间安全意识,掌握扎实的理论知识、
培养多方面的实践能力,为网络空间安全的发展添砖加瓦。
%
\end{document}
