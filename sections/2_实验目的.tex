\documentclass[../main.tex]{subfiles}
\graphicspath{{\subfix{../figures/}}}

\begin{document}
\section{实验目的}
技术是一个不断迭代和更新的领域,随着技术的更新,系统的易用性和安全性也在不断提升。然而,现有的系统为了保持稳定性,有时缺乏定期的更新维护(尤其是嵌入式系统),这给了潜在的入侵者可乘之机。入侵者可以利用已知的漏洞对旧系统进行攻击,进行破坏、窃取信息等恶意行为,从而危及用户对系统的信任和安全。

本文重点关注无线网络安全领域,针对某一特定版本的无线网络通信协议存在的漏洞展开讨论。利用嵌入式系统来模拟攻击(\texttt{Deauthentication
Flood Attack}),对该无线网络通信协议中的漏洞进行复现和深入研究。这样的实验和研究有助于揭示潜在的安全风险,并为加强无线网络的安全性提供有益的参考和建议。

通过对无线网络通信协议中的漏洞进行复现和研究,我们可以深入了解潜在的安全隐患以及可能面临的攻击方式。这种研究可以帮助我们识别系统中的薄弱环节,并采取相应的安全防护措施,以提高系统的抵御能力和安全性。

在当前信息技术高速发展的背景下,网络安全问题日益突出,尤其是在无线网络领域。随着物联网和移动通信的普及,无线网络的安全性愈发重要。对无线网络通信协议中的漏洞进行研究,不仅有助于保障用户和数据的安全,也是推动整个行业技术发展的重要一步。

通过加强对无线网络安全的研究和探索,我们可以不断提升系统的安全性和稳定性,确保用户信息的保密性和完整性。唯有如此,我们才能更好地应对日益复杂的网络安全威胁,建立一个更加安全可靠的数字化世界。
\end{document}
